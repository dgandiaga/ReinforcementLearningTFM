%
% Appendix A
%

%\graphicspath{{images/}}
\chapter{Definiciones generales}
\label{apx:ApdxA}

Para una mejor comprensi�n de los t�rminos usados en el presente trabajo. A continuaci�n se definen e ilustran una serie de conceptos usandos en el campo de los helic�pteros:\\



{\bfseries Sistema de Posicionamiento Global (GPS):}  Sistema de Posicionamiento Global el cual permite determinar en todo el mundo la posici�n de una persona, un veh�culo o una nave, con una precisi�n de entre cuatro metros y quince metros. El GPS funciona mediante una constelaci�n de 24 sat�lites artificiales uniformemente distribuidos en un total de 6 �rbitas, de forma que hay 4 sat�lites por �rbita. Esta configuraci�n asegura que siempre puedan {\em verse} al menos 8 sat�lites desde casi cualquier punto de la superficie terrestre. Los sat�lites GPS orbitan la Tierra a una altitud de unos 20.000 km y recorren dos �rbitas completas cada d�a. Describen un tipo de �rbita tal que {\em salen} y se {\em ponen} dos veces al d�a. Cada sat�lite transmite se\~nales de radio a la Tierra con informaci�n acerca de su posici�n y el momento en que se emite la se\~nal. Podemos recibir esta informaci�n con receptores GPS(receptores GPS), que decodifican las se\~nales enviadas por varios sat�lites simult�neamente y combinan sus informaciones para calcular su propia posici�n en la Tierra, es decir sus coordenadas de latitud y longitud (latitud y longitud) con una precisi�n de unos 4-10 metros.\\


{\bfseries Unidad de Medida Inercial (IMU):} Unidad de medida inercial es el nombre dado al conjunto de sensores que aportan informaci\'on sobre la aceleraci\'on y \'angulos de orientaci\'on del sistema. Los sensores m\'as comunmente empleados son aceler\'ometros y gir�scopos. La dispocisi�n de los aceler�metros son ortogonales entre ellos, de manera similar se disponen los gir�scopos.\\

{\bfseries Gradient descent (ascent)} es un algoritmo de optimizaci�n que toma los m�nimos locales de una funci�n en un punto, en pasos que son proporcionales a el negativo del gradiente (o su aproximaci�n) de la funcion este punto. Si en lugar se toman pasos proporcionales a el gradiente y se usan los maximos locales, el procedimiento se denomina {\bf Gradient ascent.}\\

{\bfseries ANN (Artificial Neural Network)} 
En ciencias de la computaci�n y afines, las redes neuronales artificiales (ANN, por sus siglas en ingl�s) son modelos computacionales inspirados en el sistema nervioso central de los animales (en particular, el cerebro), empleadas en campos como el \textit{Machine Learning} o el reconocimiento de patrones (\textit{Pattern Recognition}). Las redes neuronales artificiales se presentan generalmente como sistemas de \lq\lq neuronas\rq\rq\ interconectados que pueden calcular valores de salida a partir de los valores de entada (\textit{inputs}).\\


%
% End of Appendix A
%


%
% Appendix B
%
%\chapter{Titulo para el apendice B}
%\label{apx:ApdxB}

%
% End of Appendix B
%


%
% Appendix C
%
%\chapter{Desarrollos Matem�ticos}
%\label{apx:ApdxC}
%
%A continuaci�n se exponen las f�rmulas empleadas durante el desarrollo del presente Trabajo Fin de M�ster:
 



%
% End of Appendix C
%

%
% Appendix D
%
